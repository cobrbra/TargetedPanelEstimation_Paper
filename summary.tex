\documentclass[12pt]{article}
\textwidth 16.5cm
\textheight 23.5cm
\oddsidemargin 0pt
\topmargin -2cm
\usepackage{epsf}
%\usepackage{enumerate}
%\usepackage{natbib}
%\usepackage{url} % not crucial - just used below for the URL 

%\pdfminorversion=4
% NOTE: To produce blinded version, replace "0" with "1" below.



%\setlength{\parindent}{.3in}
%\setlength{\parskip}{.05in}
\usepackage{latexsym,amsmath,amssymb,amsfonts,amsthm,bbm,mathrsfs,breakcites,dsfont,xcolor}
\usepackage{stmaryrd,epsf}
\usepackage{natbib}
\bibliographystyle{abbrvnat}

\setcitestyle{authoryear, open={(},close={)}}
\usepackage{soul}
\usepackage{url}
\usepackage{dsfont}
\usepackage{enumerate}
\usepackage{graphicx}
\usepackage{graphics}
\usepackage{psfrag}
\usepackage{caption, subcaption}
\usepackage{multirow}
%\usepackage{algorithm}
\usepackage[ruled,vlined]{algorithm2e}

% \usepackage[backend = biber, natbib,
    % style = authoryear]{biblatex}

%\usepackage{comment}
%\newcommand{\indep}{\rotatebox[origin=c]{90}{$\models4[]}
\usepackage[colorlinks,linkcolor=black,citecolor=blue,urlcolor=blue,breaklinks = true]{hyperref}

\usepackage[acronym, toc]{glossaries-extra}

\setabbreviationstyle[acronym]{long-short}

\glssetcategoryattribute{acronym}{nohyperfirst}{true}

\renewcommand*{\glsdonohyperlink}[2]{%
 {\glsxtrprotectlinks \glsdohypertarget{#1}{#2}}}

\newtheorem{theorem}{Theorem}
\newtheorem{lemma}[theorem]{Lemma}
%\newtheorem{example}[]{Example}
\newtheorem{proposition}[theorem]{Proposition}
\newtheorem{corollary}[theorem]{Corollary}
\newtheorem{assumption}{A\!\!}
\newtheorem{example}{Example}

\DeclareMathOperator*{\argmax}{arg\,max}
\DeclareMathOperator*{\argmin}{arg\,min}
% \renewcommand{\baselinestretch}{1.25}
\newcommand{\indep}{\rotatebox[origin=c]{90}{$\models$}}
\newcommand{\red}[1]{\textcolor{red}{#1}}
\newcommand{\blue}[1]{\textcolor{blue}{#1}}
\newcommand{\green}[1]{\textcolor{green}{#1}}
\newcommand{\orange}[1]{\textcolor{orange}{#1}}
\newcommand{\var}{\mathrm{Var}}
\newcommand{\cov}{\mathrm{Cov}}
\newcommand{\bbE}{\mathbb{E}}

\usepackage{xcolor}
\usepackage[draft,inline,nomargin,index]{fixme}
\fxsetup{theme=color,mode=multiuser}
\FXRegisterAuthor{tc}{atc}{\color{red} TC}
\FXRegisterAuthor{jb}{abc}{\color{green} JB}



\makeglossaries 

\newacronym{lasso}{LASSO}{Least Absolute Shrinkage and Selection Operator}
\newacronym{tmb}{TMB}{Tumour Mutation Burden}
\newacronym{tib}{TIB}{Tumour Indel Burden}
\newacronym{icb}{ICB}{Immune Checkpoint Blockade}
\newacronym{ici}{ICI}{Immune Checkpoint Inhibitor}
\newacronym{msi}{MSI}{Micro-Satellite Instability}
\newacronym{ctla4}{CTLA-4}{Cytotoxic T Lymphocyte Associated protein 4}
\newacronym{pdl1}{PD-L1}{Programmed Death Ligand 1}
\newacronym{wes}{WES}{Whole Exome Sequencing}
\newacronym{ctdna}{ctDNA}{Circulating Tumour DNA}
\newacronym{bmr}{BMR}{Background Mutation Rate}
\newacronym{nsclc}{NSCLC}{Non-Small Cell Lung Cancer}
\newacronym{auprc}{AUPRC}{area under the precision-recall curve}
\newacronym{ectmb}{ecTMB}{Estimation and Classification of Tumour Mutation Burden}

\title{Data-driven design of targeted gene panels for estimating immunotherapy biomarkers}
 \author{Jacob R. Bradley and Timothy I. Cannings
 \\ \emph{School of Mathematics, University of Edinburgh}}
\date{}
\begin{document}


\maketitle


\begin{abstract}
We introduce a novel data-driven framework for the design of targeted gene panels for estimating exome-wide biomarkers in cancer immunotherapy. Our first goal is to develop a generative model for the profile of mutation across the exome, which allows for gene- and variant type-dependent mutation rates. Based on this model, we then propose a new procedure for estimating biomarkers such as \acrlong{tmb} and \acrlong{tib}.  Our approach allows the practitioner to construct a targeted gene panel of a prespecified size, alongside an estimator that only depends on the selected genes, which facilitates cost-effective prediction.  Alternatively, the practitioner may apply our method to make predictions based on an existing gene panel, or to augment a gene panel to a given size. We demonstrate the excellent performance of our proposal using an annotated mutation dataset from 1144 \acrlong{nsclc} patients. 



\textbf{Keywords: cancer, gene panel design, tumour indel burden, tumour mutation burden.}
\end{abstract}

It has been understood for a long time that cancer, a disease occurring in many distinct tissues of the body and giving rise to a wide range of presentations, is initiated and driven by the accumulation of mutations in a subset of a person's cells. Since the discovery of \gls{icb}\footnote{For their work on \gls{icb}, James Allison and Tasuku Honjo received the 2018 Nobel Prize for Physiology/Medicine.},  there has been an explosion of interest in cancer therapies targeting immune response and \gls{icb} therapy is now widely used in clinical practice.  \gls{icb} therapy works by targeting natural mechanisms (such as the proteins \gls{ctla4} and \gls{pdl1}) to disengage the immune system. Inhibition of these \emph{checkpoints} can promote a more aggressive anti-tumour immune response, and in some patients this leads to long-term remission . However, \gls{icb} therapy is not always effective and may have adverse side-effects, so determining which patients will benefit in advance of treatment is vital. 


Exome-wide prognostic biomarkers for immunotherapy are now well-established -- in particular, \gls{tmb} is used to predict response to immunotherapy.  \gls{tmb} is defined as the total number of non-synonymous mutations occurring throughout the tumour exome, and can be thought of as a proxy for how easily a tumour cell can be recognised as foreign by immune cells. However, the cost of measuring \gls{tmb} using \gls{wes} currently prohibits its widespread use as standard-of-care.  Sequencing costs, both financial and in terms of the time taken for results to be returned, are especially problematic in situations where high-depth sequencing is required, such as when utilising blood-based \gls{ctdna} from liquid biopsy samples. The same issues are encountered when measuring more recently proposed biomarkers such as \gls{tib}, which counts the number of frameshift insertion and deletion mutations. There is, therefore, demand for cost-effective approaches to estimate these biomarkers.

In this paper we propose a novel, data-driven method for biomarker estimation, based on a generative model of how mutations arise in the tumour exome.  More precisely, we model mutation counts as independent Poisson variables, where the mean number of mutations depends on the gene of origin and variant type, as well as the \gls{bmr} of the tumour. Due to the ultrahigh-dimensional nature of sequencing data, we use a regularisation penalty when estimating the model parameters, in order to reflect the fact that in many genes' mutations arise purely according to the \gls{bmr}. In addition, this identifies a subset of genes that are mutated above or below the background rate. Our model facilitates the construction of a new estimator of \gls{tmb}, based on a weighted linear combination of the number of mutations in each gene. The vector of weights is chosen to be sparse (i.e.~have many entries equal to zero), so that our estimator of \gls{tmb} may be calculated using only the mutation counts in a subset of genes. In particular, this allows for accurate estimation of \gls{tmb} from a targeted gene panel, where the panel size (and therefore the cost) may be determined by the user.  We demonstrate the excellent practical performance of our framework using a \gls{nsclc} dataset, and include a comparison with the existing state-of-the-art data-driven approaches for estimating \gls{tmb}.  Moreover, since our model allows variant type-dependent mutation rates, it can be adapted easily to predict other biomarkers, such as \gls{tib}. Finally, our method may also be used in combination with an existing targeted gene panel; we can estimate the biomarker directly from that panel, or first augment the panel and then construct an estimator.  

Due to its emergence as a biomarker for immunotherapy in recent years, a variety of groups have considered methods for estimating \gls{tmb}. A simple and common way to estimate \gls{tmb} is via the proportion of mutated codons in a targeted region. \citet{budczies_optimizing_2019} investigate how the accuracy of predictions made in this way are affected by the size of the targeted region, where mutations are assumed to occur at uniform rate throughout the genome. More recently \citet{yao_ectmb_2020} modelled mutations as following a negative binomial distribution while allowing for gene-dependent rates, which are inferred by comparing nonsynonymous and synonymous mutation counts. In contrast, our method does not require data including synonymous mutations. Linear regression models have been used for both panel selection \citep{lyu_mutation_2018} and for biomarker prediction \citep{guo_exon_2020}. A review of some of the issues arising when dealing with targeted panel-based predictions of \gls{tmb} biomarkers is given by \citet{wu_designing_2019}. Finally, we are unaware of any methods for estimating \gls{tib} from targeted gene panels. 

\bibliography{zotero-refs.bib}


\end{document}